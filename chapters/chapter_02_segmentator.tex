\chapter{Segmentator}

\section{Abstract}
High-resolution (functional) magnetic resonance imaging (MRI) at ultra high magnetic fields (7 Tesla and above) enables researchers to study how anatomical and functional properties change within the cortical ribbon, along surfaces and across cortical depths. These studies require an accurate delineation of the gray matter ribbon, which often suffers from inclusion of blood vessels, dura mater and other non-brain tissue. Residual segmentation errors are commonly corrected by browsing the data slice-by-slice and manually changing labels. This task becomes increasingly laborious and prone to error at higher resolutions since both work and error scale with the number of voxels. Here we show that many mislabeled, non-brain voxels can be corrected more efficiently and semi-automatically by representing three-dimensional anatomical images using two-dimensional histograms. We propose both a uni-modal (based on first spatial derivative) and multi-modal (based on compositional data analysis) approach to this representation and quantify the benefits in 7 Tesla MRI data of nine volunteers. We present an openly accessible Python implementation of these approaches and demonstrate that editing cortical segmentations using two-dimensional histogram representations as an additional post-processing step aids existing algorithms and yields improved gray matter borders. By making our data and corresponding expert (ground truth) segmentations openly available, we facilitate future efforts to develop and test segmentation algorithms on this challenging type of data.

% \papercitation{Schneider2019}{Printed}

\section{Introduction}
\lipsum[1-1] 

\section{Methods}
\lipsum[1-2]
\cite{Lakatos1970}

\section{Results}
\lipsum[1-3]

\section{Discussion}
\lipsum[1-4]

\section{Conclusion}
\lipsum[1-5]

\clearpage
\printbibliography[heading=subbibnumbered, title={References}]