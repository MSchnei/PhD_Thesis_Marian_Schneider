\chapter{Motion dependent pRF}

\section{Abstract}
Motion signals can bias the perceived position of visual stimuli. Electro-physiological studies have shown that while the apparent position of a stimulus is biased in the direction of motion, the receptive field (RF) of neurons is shifted in the direction opposite to motion, at least in cats and macaque monkeys. In humans, functional magnetic resonance imaging (fMRI) studies have shown direction- and speed-dependent effects of motion on population RF (pRF) measures. However, the direction in which motion displaces the pRF remains unclear. We addressed this question by measuring pRFs and systematically varying two factors: the motion direction of the carrier pattern (3 levels: inward, outward and flicker motion) and the contrast of the mapping stimulus (2 levels: low and high stimulus contrast). We found that pRFs in early visual cortex were shifted against the direction of motion for low-contrast stimuli but not for high stimulus contrast. Correspondingly, we observed that the low-contrast, but not the high-contrast, stimuli led to systematic perceptual displacements in the direction of motion. We offer an explanation in form of a model for why apertures are perceptually shifted in the direction of motion even though pRFs shift in the opposite direction.

% \papercitation{Schneider2019}{Printed}

\section{Introduction}
\lipsum[1-1]

\section{Methods}
\lipsum[1-2]

\section{Results}
\lipsum[1-3]

\section{Discussion}
\lipsum[1-4]
\cite{Lakatos1970}

\section{Conclusion}
\lipsum[1-5]

\clearpage
\printbibliography[heading=subbibnumbered, title={References}]