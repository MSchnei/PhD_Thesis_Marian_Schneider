\chapter{Motion quartet}

\section{Abstract}
The specific contents of human consciousness rely on the activity of specialized neurons in cerebral cortex. We hypothesized that the conscious experience of a specific visual motion axis is reflected in response amplitudes of direction-selective clusters in the human motion complex. Using sub-millimeter functional magnetic resonance imaging at ultra-high field (7 Tesla) we identified fine-grained clusters that were tuned to either horizontal or vertical motion presented in an unambiguous motion display. We then recorded their responses while human observers reported the perceived axis of motion for an ambiguous apparent motion display. Although retinal stimulation remained constant, subjects reported recurring changes between horizontal and vertical motion percepts every 7 to 13 seconds. We found that these perceptual states were dissociatively reflected in the response amplitudes of the identified horizontal and vertical clusters. We also found that responses to unambiguous motion were organized in a columnar fashion such that motion preferences were stable in the direction of cortical depth and changed when moving along the cortical surface. We suggest that activity in these specialized clusters is involved in tracking the distinct conscious experience of a particular motion axis.

% \papercitation{Schneider2019}{Printed}

\section{Introduction}
\lipsum[1-1]

\section{Methods}
\lipsum[1-2]

\section{Results}
\lipsum[1-3] 
\cite{Lakatos1970}

\section{Discussion}
\lipsum[1-4]

\section{Conclusion}
\lipsum[1-5]

\clearpage
\printbibliography[heading=subbibnumbered, title={References}]