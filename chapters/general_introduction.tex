\chapter{Chapter 1 - General introduction}

\section*{Abstract}
\lipsum[2-2]

\papercitation{MTTHarms2017}{Printed}

\section{Introduction}

early- and mid-level visual areas
high-resolution
conscious perception
visual motion perception

\section{motivation}
The candidate’s primary intrinsic motivation for pursuing a PhD is to better understand the relationship between visual awareness and the human brain. Ample of empirical evidence suggests that perception is related to the brain’s activity. However, it remains largely elusive exactly which neural mechanisms bring about the contents of perceptual experiences (e.g. to bring about the experience of seeing a red rose).

Progress in science has often been preceded by jumps in methodological development. The motion of stars was only understood after we were able to systematically observe them following the invention of the telescope. In analogy, better understanding the relationship between brain activity and perception likely requires the advance of current imaging methods. Thus, a secondary motivation for this PhD is to further widen the scope of applications in fMRI research.

\section{A brief history of this thesis}
The present work was made possible by a generous PhD grant of the Netherlands Organization (NWO). And I should emphasize, in advance, that I am very grateful to the NWO for their funding. 

I cannot resist the temptation to share a few details of the proposal.

The grant proposal was given the rather presumptuous title "The pillars of perception". The underlying idea was that perception was constructed like a Greek Temple. Just as the Parthenon in the Acropolis, so perception was built from building blocks that are arranged in columns and layers. The only difference being that the 
building blocks of perception were not made out of schist and lime stone, like the acropolis, but out of neuronal matter. Looking back, I am not sure which one is more striking: my groundless naivety or the fact that it actually received funding.

I was hoping to identify the "Columnar Correlates of Consciousness". I still remember, that when I presented my research ideas in the first progress report, with quite a bit of enthusiasm, that Elia Formisano dryly pointed out that I had manage to combine three very contested concepts (columns, correlation, consciousness) in just a single sentence.

Of course, like several other white men on this planet (insert ref to Tononi, van Mourik, Anil Seth), I like to see myself in the tradition of "studying the universe within". The tale goes as follows.

\section{Background}
The present chapter discusses methods and research findings that served as background for these studies. Specifically, acoustic properties of natural sounds are introduced, and the anatomical and functional properties of subcortical and cortical auditory structures are outlined

\subsection{Search for NNC}
Main motivation for the present work is the search for the neural correlates of consciousness (NCC) and in particular the content-specific NCC.

- Chalmers: What is a Neural Correlate of Consciousness?
" What does it mean to be a neural correlate of consciousness? At first glance, the answer might seem to be so obvious that the question is hardly worth asking. An NCC is just a neural state that directly correlates with a conscious state, or which directly generates consciousness, or something like that. One has a simple image: when your NCC is active, perhaps, your consciousness turns on, and in a corresponding way."

"Arguably the most interesting states of consciousness are specific states of consciousness: the fine-grained states of subjective experience that one is in at any given time. Such states might include the experience of a particular visual image, of a particular sound pattern, of a detailed stream of conscious thought, and so on. A detailed visual experience, for example, might include the experience of certain shapes and colors in one’s environment, of specific arrangements of objects, of various relative distances and depths, and so on."

Koch, Massimini, Boly, Tononi:
"To identify the content-specific NCC, neural activity when a particular stimulus (such as a face) is perceived is compared with,
neural activity when that stimulus is not perceived with the sensory stimulus and the overall state of the par- ticipant kept constant under both circumstances"


\subsection{Dissociating perceptual from retinal properties}
retinal properties of the stimulus become dissociated from the conscious percept:

An important goal of neuroscience research is to dissociate neural activity pertaining to perceptual processes from that merely reflecting the sensory information from the visual scene. This can be
achieved by exposing the visual system to stimuli that allow for more than one percept despite invariant visual input.

using stimuli that allow for distinguishing perceptual and stimulus-related neuronal activity

First study our approach was:
1. identify areas responding to stimulus features (eg. horizontal and vertical)
2. measure their signal while participant is exposed to unchanging stimulus that gives rise to two percepts

Second study our approach was:
1: The physical and retinal locations of the drifting Gabors were identical
2: perceptually they were shifted

\subsection{High-resolution UHF fMRI data}
- discuss specific challenges and opportunities presented by UHF-fMRI
Polimeni, J. R., Renvall, V., Zaretskaya, N., & Fischl, B. (2018). Analysis strategies for high-resolution UHF-fMRI data. NeuroImage, 168(March), 296–320. https://doi.org/10.1016/j.neuroimage.2017.04.053

Dumoulin, S. O., Fracasso, A., van der Zwaag, W., Siero, J. C. W., & Petridou, N. (2018). Ultra-high field MRI: Advancing systems neuroscience towards mesoscopic human brain function. NeuroImage, 168(March), 345–357. https://doi.org/10.1016/j.neuroimage.2017.01.028

Kemper, V. G., De Martino, F., Emmerling, T. C., Yacoub, E., & Goebel, R. (2018). High resolution data analysis strategies for mesoscale human functional MRI at 7 and 9.4T. NeuroImage, 164(January), 48–58. https://doi.org/10.1016/j.neuroimage.2017.03.058

De Martino, F., Yacoub, E., Kemper, V., Moerel, M., Uludag, K., De Weerd, P., … Formisano, E. (2018). The impact of ultra-high field MRI on cognitive and computational neuroimaging. NeuroImage, 168(March), 366–382. https://doi.org/10.1016/j.neuroimage.2017.03.060


- Functional MRI (fMRI) benefits from both increased sensitivity and specificity with increasing magnetic field strength
- utilize the dramatic increases in sensitivity and specificity to acquire high-resolution data reaching sub-millimeter scales
which enable new classes of experiments to probe the functional organization of the human brain'





\subsection{Functional localizers}
- functional localizer debate (Kanwisher, Friston)
- double dipping (Kriegeskorte)
- localizing hMT+
- localizing early visual areas (pRF mapping)
- show figure of delineated hMT+
- show figure of delineated early visual areas





\section{Challenges and goals}
\lipsum[2-5]

\section{Purpose and outline of the thesis}

Research aims and outline of the thesis

The first part presents research embedded within the neuroscience of vision and relies on advances with regard to ultra-high field fMRI and computational neuroimaging (Wandell & Winawer, 2015) in order to perform comparative theory testing ...

The second part presents research aimed at understanding structure-function relationships in the brain. It exploits advances in parallel...