%%% Defines the packages for your thesis

% Set font families 
% Palatino for rm and math | Helvetica for ss | Courier for tt
\usepackage{mathpazo} % math & rm
\usepackage[scaled]{helvet} % ss
\usepackage{courier} % tt
\normalfont
\usepackage[T1]{fontenc}

%% For using utf8 symbols in the text
\usepackage[utf8]{inputenc}

%% For some fake text, can be removed if you have some content 
\usepackage{blindtext}
\usepackage{lipsum}  

%% For adding illustrations and graphics
\usepackage{graphicx}

%% For adding some demonstration text
\usepackage{lipsum}

%% For avoiding orphans (sole lines at the bottom of the page) and widows ( sole lines at the top of the page)
\usepackage[all]{nowidow}

%% For defining custom colors with hex code
\usepackage{xcolor}
%% Define custom colors for links
\definecolor{customgreen}{HTML}{000000}%006d2c
\definecolor{customblue}{HTML}{000000}%084081
\definecolor{customred}{HTML}{000000}%bd0026

%% For linking figures, sections, websites
\usepackage{hyperref}
\hypersetup{
    colorlinks=true,
    linkcolor=customblue,
    filecolor=customblue,      
    urlcolor=customred,
    citecolor=customgreen,
    linktocpage=true,
}

%% For citations with biblatex
\usepackage[american]{babel}
\usepackage{epstopdf}
\usepackage{csquotes}
\usepackage[
    style=apa,
    backend=biber,
    sortcites=true,
    sorting=nyt,
    isbn=false,
    url=false,
    doi=false,
    eprint=false,
    hyperref=true,
    backref=false,
    giveninits=false,
    refsection=chapter,
    uniquename=false,
]{biblatex}

% declare language mapping (important for apa style to work)
\DeclareLanguageMapping{american}{american-apa}

%% Import bibliography file
\addbibresource{bibliography.bib}

%% better looking tables by default
\usepackage{booktabs}

%% a different table class
\usepackage{tabularx,ragged2e}
\newcolumntype{C}{>{\Centering\arraybackslash}X} % centered "X" column
\newcolumntype{L}{>{\arraybackslash}X} % left aligned "X" column

%% allows you to rotate figures and tables
\usepackage[figuresright]{rotating}

%% for using things like \mathbb
\usepackage{amssymb}

%% for other math environments
\usepackage{amsmath}
\DeclareMathOperator*{\argmin}{argmin}
\DeclareMathOperator{\arctantwo}{arctan2}

%% for automatically breaking equations over lines
\usepackage{breqn}

%% for hyphenation of works already containing an hyphen
%% use \hyp{} in those words, e.g. Levenberg\hyph{}Marquardt
\usepackage{hyphenat}

%% for more advanced appendices
\usepackage{appendix}

%% for referencing things by name
\usepackage{nameref}
