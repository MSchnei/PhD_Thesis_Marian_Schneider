%%% Defines the packages for your thesis will be using

%% The font used
\usepackage{palatino}

%% For using utf8 symbols in the text
\usepackage[utf8]{inputenc}

%% For some fake text, can be removed if you have some content 
\usepackage{blindtext}
\usepackage{lipsum}  

%% For adding illustrations and graphics
\usepackage{graphicx}

%% For adding some demonstration text
\usepackage{lipsum}  

%% For mini TOC in each chapter
\usepackage{minitoc}

%% For more advanced citations
\usepackage[citestyle=authoryear,bibstyle=authoryear,refsection=chapter, doi=false,isbn=false,url=false]{biblatex}
\addbibresource{bibliography.bib}

%% better urls in bibliography
\usepackage[hidelinks]{hyperref}

%% better looking tables by default
\usepackage{booktabs}

%% a different table class
\usepackage{tabularx,ragged2e}
\newcolumntype{C}{>{\Centering\arraybackslash}X} % centered "X" column
\newcolumntype{L}{>{\arraybackslash}X} % left aligned "X" column

%% allows you to rotate figures and tables
\usepackage[figuresright]{rotating}

%% for using things like \mathbb
\usepackage{amssymb}

%% for other math environments
\usepackage{amsmath}
\DeclareMathOperator*{\argmin}{argmin}
\DeclareMathOperator{\arctantwo}{arctan2}

%% for automatically breaking equations over lines
\usepackage{breqn}

%% for hyphenation of works already containing an hyphen
%% use \hyp{} in those words, e.g. Levenberg\hyph{}Marquardt
\usepackage{hyphenat}

%% for more advanced appendices
\usepackage{appendix}

%% for referencing things by name
\usepackage{nameref}