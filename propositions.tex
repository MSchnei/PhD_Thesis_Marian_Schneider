\documentclass[a4paper]{article}
\usepackage{a4wide}
\usepackage{palatino}

\author{\large{Marian Schneider, August 02 2019}}
\title{
	\large{Propositions accompanying the PhD-thesis}\\
	\huge{Representational content in human cortical systems and visual consciousness}
}
\date{August 02 2019}

% A minimum of eight and a maximum of eleven propositions shall be added to the thesis. Four propositions must be related to the subject of the thesis. Three positions must be related to the doctoral candidate’s discipline, with the exception of the subject of the thesis. One proposition must be related to the valorisation opportunities of the subject of the thesis. Any other propositions do not have to be related to the subject of the thesis or to the doctoral candidate’s discipline. The supervisor must approve the propositions.


\begin{document}

\maketitle
\thispagestyle{empty}

    \begin{enumerate}
    	\begin{large}
    	
		%  Four propositions must be related to the subject of the thesis
		\item Neuroimaging methods in humans have often been considered ill-suited to investigate content-specific NCC, given difficulties to reliably map and track representational content for neural systems in humans
		
		\item With the introduction of UHF scanners and the development of appropriate analysis techniques, it is now possible with fMRI to map sub-categorical representational content in human cortical systems

		\item it is now furthermore possible with fMRI to formulate and test mechanisms that link these representations to conscious experience.

		\item columnar and laminar fMRI are powerful but fiddly

		%  Three positions must be related to the doctoral candidate’s discipline, with the exception of the subject of the thesis.
		\item A small minority of researchers worries too much about reproducibility in neuroimaging, while the majority of researchers pays too little attention to it.

		\item Cognitive neuroscience will be subsumed by cognitive computational neuroscience

		\item proposition 3 out of 3 related to the doctoral candidate’s discipline
		Discplines and displinary borders are a things of the past.

        % One proposition must be related to the valorisation opportunities of the subject of the thesis.
		\item proposition 1 out of 1 related to the valorisation opportunities of the subject of the thesis
		Understanding consciousness and being able to manipulate and to create artifical instances of it affords endless opportunities for valorization and poses equally a lot of risks for society.  

        % Any other propositions do not have to be related to the subject of the thesis or to the doctoral candidate’s discipline.
        \item proposition 1 out of 3 voluntary propositions
        Creating systems and processes that are conscious is attainable within the lifetime of the doctoral candidate.

        \item proposition 2 out of 3 voluntary propositions
        It is not the primary task of the scientist to worry about the implications of artifical consciousness.

        \item proposition 3 out of 3 voluntary propositions
        the free energy minimization paradigm is general enough a framework about brain function (and life in general) that it would be surprising if it did not hold the key to explaining consciousness


	    \end{large}
    \end{enumerate}
\end{document}

