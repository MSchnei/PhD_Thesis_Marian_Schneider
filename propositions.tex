\documentclass[a4paper]{article}
\usepackage{a4wide}
\usepackage{palatino}

\author{\large{Marian Schneider, August 02 2019}}
\title{
	\large{Propositions accompanying the PhD-thesis}\\
	\huge{Representational content in human cortical systems and visual consciousness}
}
\date{August 02 2019}

% A minimum of eight and a maximum of eleven propositions shall be added to the thesis. Four propositions must be related to the subject of the thesis. Three positions must be related to the doctoral candidate’s discipline, with the exception of the subject of the thesis. One proposition must be related to the valorisation opportunities of the subject of the thesis. Any other propositions do not have to be related to the subject of the thesis or to the doctoral candidate’s discipline. The supervisor must approve the propositions.


\begin{document}

\maketitle
\thispagestyle{empty}

    \begin{enumerate}
    	\begin{large}
    	
		%  Four propositions must be related to the subject of the thesis
		\item Most researchers in the consciousness community are sceptical about the suitability of human neuroimaging methods for investigating the neural correlates of (visual) conscious content.
		
		\item This scepticism results from longstanding difficulties to reliably map and track the representational content of neural systems in humans.
		
		\item With the introduction of high field scanners and the development of appropriate analysis techniques, it is now possible with fMRI to map (sub-categorical) representational content in human cortical systems and to relate it to contents in consciousness.

		\item It is now possible with fMRI to formulate and test mechanistic descriptions that link representational content in human cortical systems to content in consciousness.

		%  Three positions must be related to the doctoral candidate’s discipline, with the exception of the subject of the thesis.
		\item Columnar and laminar fMRI are potentially powerful tools for the investigation of brain function but principled obstacles remain to exploit their full potential.

		\item A small minority of researchers in neuroimaging worries excessively about reproducibility, while the majority of researchers pays too little attention to the problem.		
		
		\item Cognitive neuroscience will be subsumed by cognitive computational neuroscience in the long run.

        % One proposition must be related to the valorisation opportunities of the subject of the thesis.
		\item Being able to manipulate and to create artificial instances of consciousness will afford endless opportunities for valorization and poses equally a lot of risks for society.  

        % Any other propositions do not have to be related to the subject of the thesis or to the doctoral candidate’s discipline.
        \item Creating systems and processes that are conscious is attainable within the lifetime of the doctoral candidate.

        \item The free energy minimization paradigm (or, at least, a flavor of predictive processing) will become the prevailing paradigm to interpret studies of brain function and structure in the next 20 years.

	    \end{large}
    \end{enumerate}
\end{document}

