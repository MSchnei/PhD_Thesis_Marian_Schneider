\chapter{Summary}
Functional magnetic resonance imaging (fMRI) research at ultra-high field (UHF) strength offers new opportunities and challenges for the study of the neural correlates of visual conscious content in humans (content-specific NCC). The goal of this thesis was to leverage these opportunities and to overcome some of the challenges. One such challenge is tissue class segmentation of UHF images. In Chapter~\ref{ch:chapter02}, we describe and implement a method that removes non-brain voxels efficiently and semi-automatically by representing three-dimensional anatomical images in a two-dimensional histogram. We demonstrate that this method improves cortical grey matter definitions when used as a post-processing step to existing segmentation algorithms and frees a considerable amount of researcher time that would otherwise need to be spent on manual corrections. 

Equipped with this new tool, we went on to perform two high-field high-resolution fMRI experiments to clarify how representations in cortical structures are related to the perception of apparent motion and apparent position. In Chapter~\ref{ch:chapter03}, we study how responses in columnar clusters of the human motion complex (hMT+) relate to the conscious experience of a specific visual motion axis. Leveraging the increased sensitivity of UHF, we first map out distinct, fine-grained clusters that either preferred horizontal or vertical motion. We then measure their response modulations during ambiguous stimulation with the bistable motion quartet. We find that when participants indicate the conscious percept of horizontal motion the response amplitude increases in the "horizontal" cluster and decreases in the "vertical" cluster, while we find the reverse pattern during vertical motion. In addition, we demonstrate that these clusters are organized in a columnar fashion such that preferences for vertical or horizontal motion are stable in the direction of cortical depth and change when moving along the cortical surface. These results support the idea that hMT+ makes up part of the content-specific NCC for visual motion and, more importantly, extend this idea by demonstrating involvement of specific horizontal and vertical columnar clusters in tracking conscious experience of a particular motion axis.

While this study imposed important constraints on the location of content-specific NCC, it did not offer a strong constraint on mechanisms that link cortical activity to conscious visual perception. In Chapter~\ref{ch:chapter04}, we therefore explore the ability of the population receptive field (pRF) mapping paradigm to connect responses in early and mid-level visual cortex to the perceived position of visual stimuli. Using psychophysics, we first identify stimulus contrast as an important factor that influences the strength of motion-induced position shifts. We observe that at high stimulus contrast, when sufficient sensory evidence is available, little to no perceptual displacements occur, while at low stimulus contrast, when signal uncertainty is increased, substantial perceptual displacements occur. Correspondingly, we find that pRF centers are substantially shifted against the direction of motion for low-contrast, but not high-contrast, stimuli. Based on these observations, we propose a model that links pRF estimates to perceived stimulus position. This model is based on the idea that early visual areas send shifted population codes in response to motion stimuli that bias the conscious percept of stimulus position in the direction of motion.

Neuroimaging methods in humans have often been considered ill-suited to investigate content-specific NCC, given difficulties to reliably map and track representational content for neural systems in humans. This thesis can be understood as a demonstration that, with the introduction of UHF scanners and the development of appropriate analysis techniques, it is now possible with fMRI to map (sub-categorical) representational content in human cortical systems and to formulate and test mechanisms that link these representations to conscious experience.