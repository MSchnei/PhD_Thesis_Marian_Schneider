\chapter{Summary}
Existing knowledge of how cortical responses link to conscious content in humans is either inferred from animal models or from human studies limited by lower spatial resolution. While previous studies could relate distinct categorical percepts (faces vs. places) to signal differences across brain areas, measuring responses at sub-millimeter resolution allowed us to link sub-category conscious percepts (vertical vs. horizontal motion) to amplitude changes of separate populations within the same brain area. Furthermore, preferences for horizontal and vertical motion were organized into columnar clusters. We pave the way for future studies investigating if columnar clusters represent sub-categorical distinctions in conscious content different from motion or in high-level perceptual and cognitive phenomena.

As a result, there is an unsatisfactory gap between human psychophysics, animal electro-physiology, and human fMRI neuroimaging. Humans experience a well-established shift in perceived position during the MIPS illusion. When cats and monkeys are presented with stimuli that elicit MIPS in humans, RF shifts of single V1 and V4 neurons are observed \parencite{Fu2004, Sundberg2006}. At the same time, for humans no corresponding change in retinotopic representation in V1 -V3 has been found \parencite{Liu2006} and the specific effects of motion direction on pRF properties could not be clarified \parencite{Harvey2016}.